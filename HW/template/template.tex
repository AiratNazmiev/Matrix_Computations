\documentclass{article}
\usepackage[T2A]{fontenc}
\usepackage[utf8]{inputenc}
\usepackage[english,russian]{babel}
\usepackage{amsmath,amssymb}
\usepackage{amsfonts}
\usepackage{xcolor}
\usepackage{enumitem}
\usepackage[top=2cm, bottom=4.5cm, left=2.5cm, right=2.7cm]{geometry}
\usepackage{lastpage}
\usepackage{fancyhdr}
\usepackage{mathrsfs}
\usepackage{hyperref}

\newcommand{\rank}{\operatorname{rank}}

\newcommand{\prob}[1]{\item \textbf{(#1 баллов)}.}
\newcommand{\bonus}{\item \textbf{(Бонус)}.~}

\setlength{\parindent}{0.0in}
\setlength{\parskip}{0.05in}

\pagestyle{fancyplain}
\headheight 35pt           
\rhead{}
\chead{\textbf{\Large Домашняя работа 1} \\ (теория) \\  }
\lhead{ФКН ВШЭ \\ Основы Матричных Вычислений \\ Весенний семестр 2025} 
\lfoot{}
\cfoot{}
\rfoot{\small\thepage}
\headsep 1.5em


\begin{document}
		\begin{center}
			Работу выполнил:
            
            \textbf{Назмиев Айрат, группа ***}
		\end{center}

		\section*{Задача 1}
			\begin{equation}
				a = \frac{b}{c}, \quad c = 0.
			\end{equation}		
			Ответ $42$. 
		\section*{Задача 2}
			\begin{equation}\label{eq:wise_equation}
				\det \begin{bmatrix}
					a & b \\
					c & d
				\end{bmatrix} = a^2 - b + c
			\end{equation}		
			Согласно формуле \ref{eq:wise_equation} ответ $322$. 
			
	\newpage
	\section*{Полезные ссылки}%
	\label{sec:Полезные ссылки}
		\begin{enumerate}
			\item \href{https://www.overleaf.com/project}{онлайн \LaTeX\  компилятор}
			\item \href{https://detexify.kirelabs.org/classify.html}{рисуешь символ, получаешь формулу \LaTeX}
		\end{enumerate}	
\end{document}
