\documentclass{article}
\usepackage[T2A]{fontenc}
\usepackage[utf8]{inputenc}
\usepackage[english,russian]{babel}
\usepackage{amsmath,amssymb}
\usepackage{amsfonts}
\usepackage{xcolor}
\usepackage{enumitem}
\usepackage[top=2cm, bottom=4.5cm, left=2.5cm, right=2.7cm]{geometry}
\usepackage{lastpage}
\usepackage{fancyhdr}
\usepackage{mathrsfs}

\newcommand{\rank}{\operatorname{rank}}

\newcommand{\prob}[1]{\item \textbf{(#1 баллов)}.}
\newcommand{\bonus}{\item \textbf{(Бонус)}.~}

\setlength{\parindent}{0.0in}
\setlength{\parskip}{0.05in}

\pagestyle{fancyplain}
\headheight 35pt           
\rhead{\textbf{Версия от:} 30.01.25 \\ \textbf{Дедлайн:} 13.02.25 в 23:59}
\chead{\textbf{\Large Домашняя работа 1} \\ (теория) \\  }
\lhead{ФКН ВШЭ \\ ОМВ \\ Весенний семестр 2025} 
\lfoot{}
\cfoot{}
\rfoot{\small\thepage}
\headsep 1.5em


 
\begin{document}

\begin{enumerate}
    \item \textbf{(6 баллов: 1+2+2+1)} Пусть задан вектор $u\in\mathbb{C}^n\colon \|u\|_2=1$. Найдите все $\alpha \in \mathbb{C}$, для которых $A = I - \alpha u u^*$ является: 1) эрмитовой 2) косоэрмитовой 3) унитарной 4) нормальной. Для пункта~{3)} также нарисуйте найденные $\alpha$ на комплексной плоскости.
    \item \textbf{(16 баллов: 7+9)}
    \begin{enumerate}
        \item
    Докажите, что для любой косоэрмитовой $K\in\mathbb{C}^{n\times n}$ матрица:
    \begin{equation} \label{eq:transform}
        Q = (I - K)^{-1} (I+K)
    \end{equation}
    будет унитарной. 
    \item Найдите множество всех унитарных матриц, которые представимы в виде~\eqref{eq:transform}.
        \end{enumerate}
    \item \textbf{(14 баллов)}
        Докажите, что
    \[
       \|A\|_{1\to 2} \leq  \|A\|_{2} \leq \sqrt{n} \|A\|_{1\to 2
    }, \quad \forall A\in\mathbb{C}^{m\times n}.
    \] 
    \textbf{Указание:} В случае использования констант эквивалентности векторных норм необходимо обосновывать это значение констант. 
   \item \textbf{(16 баллов: 3+6+7)}
   Обозначим $A = \begin{bmatrix}0 & 1 \\ 0 & 0 \end{bmatrix}$ и $A_n = \begin{bmatrix} 0 & 1 \\ 1/n & 0 \end{bmatrix}, n\in\mathbb{N}$.
   \begin{enumerate}
       \item Обоснуйте сходимость $A_n\to A$, $n\to\infty$ исходя из определения сходимости с помощью норм.
       \item Найдите собственные разложения $A_n = S_n \Lambda_n S_n^{-1}$ и проверьте существование пределов для каждой из $S_n$,$\Lambda_n$ и $S_n^{-1}$. Почему не у всех из этих матриц существует предел?
       \item Найдите разложения Шура $A_n = U_n T_n U_n^{-1}$ и проверьте существование пределов для каждой из $U_n$,$T_n$ и $U_n^{-1}$.
   \end{enumerate}
    \item \textbf{(7 баллов)} Докажите, что нормальная матрица является унитарной тогда и только тогда, когда все ее собственные значения по модулю равны $1$.
    \item \textbf{(13 баллов: 6+7)}
    \begin{enumerate}
        \item Докажите, что для любой $A\in\mathbb{C}^{m\times n}$, $m\geq n$, справедливо:
    \[
        \|A\|_2 \leq \|A\|_F \leq \sqrt{n} \|A\|_2.
    \]
        \item Найдите все матрицы $A\in\mathbb{C}^{n\times n}$, удовлетворяющие $\|A\|_F = \sqrt{n} \|A\|_2$.
    \end{enumerate}
    \item \textbf{(12 баллов: 4+4+4)}
    Дана нормальная матрица $A\in\mathbb{C}^{n\times n}$ и её разложение Шура $A = U\Lambda U^*$.
    \begin{enumerate}
        \item Запишите сингулярное разложение матрицы $A$, используя матрицы $U$ и $\Lambda$.
        \item Покажите, что $\sigma_1(A) = \max_i |\lambda_i(A)|$.
        \item Приведите пример  матрицы $A \in \mathbb{C}^{2\times 2}$, не являющейся нормальной и для которой полученное в (b) выражение неверно. 
    \end{enumerate}
    \item \textbf{(16  баллов)}.
    Докажите, что для множителя $W$ (см. обозначения в лекциях) из полярного разложения матрицы $A\in\mathbb{R}^{m\times n}$, $m\geq n$ выполняется:
    \[  
        \frac{\|A^\top A - I\|_2}{\|A\|_2 + 1} \leq \|A - W \|_2 \leq \|A^\top  A - I\|_2.
    \]
    \textbf{Замечание:} Неравенство дает соотношение между двумя мерами близости (в смысле спектральной нормы) матрицы $A$ к ортогональной: $\|A^\top A - I\|$ и $\min_{Q: Q^\top Q=I} \|A - Q\|$.
\end{enumerate}

\section*{Бонусные задачи}
\begin{enumerate}
    \item \textbf{(20 б. баллов)}. 
    Назовем норму $\|\cdot\|^{*}$ двойственной к $\|\cdot\|$ над $\mathbb{R}^{m\times n}$, если 
    \[
        \|A\|^{*} = \max_{B\colon \|B\|=1} \mathrm{Tr}\left(AB^\top\right).
    \]
    Докажите, что
    \begin{enumerate}
        \item $\left(\|A\|^{*}\right)^* = \|A\|$;
        \item норма $\|\cdot\|$ унитарно-инвариантна тогда и только тогда, когда унитарно-инвариантна $\|\cdot\|^{*}$.
    \end{enumerate}
    \item \textbf{(30 б. баллов)}.
 Докажите субмультипликативность векторной $p$-нормы матрицы:
        \[
            \|A\|_{p,\mathrm{vec}} \equiv \|\mathrm{vec}(A)\|_p
        \]
        при $1\leq p \leq 2$.
     \item \textbf{(50 б. баллов)}.
     Решите задачу 8 для произвольной унитарно-инвариантной нормы $\|\cdot\|$: 
    \[  
        \frac{\|A^\top A - I\|}{\|A\|_2 + 1} \leq \|A - W \| \leq \|A^\top  A - I\|.
    \]
    Считайте известным неравенство $\|AB\|\leq \|A\|_2 \|B\|$.
\end{enumerate}
\end{document}